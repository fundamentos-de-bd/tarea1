\documentclass{article}

\usepackage[spanish]{babel}

\usepackage[margin = 1.5cm]{geometry}

\usepackage{enumitem}


\begin{document}
    \title{Fundamentos de Bases de Datos \\
        Tarea 1 \\
        Conceptos básicos} 
    \author{Díaz Gómez Silvia \\
    Eugenio Aceves Narciso Isaac \\
    Quiroz Castañeda Edgar}
    \date{18 de febrero del 2019}
    \maketitle

    \section{Conceptos generales}

    \begin{enumerate}[label=\alph*.]
        \item {
            ¿Porqué elegirías almacenar datos en un \textbf{sistema de bases 
            datos} en lugar de simplemente almacenarlos utilzizando el 
            \textbf{sistema de archivos} de un sistema operativo?¿En qué caso no
            tendría sentido utilizar un sistema de bases de datos?
        }
            {
                Elegiría el SBD si vale la pena la inversión en tiempo y recursos
                en todo el diseño e infraestructura de la relacionados. Es decir, si mis 
                necesidades involucran preservar la integridad de los datos, tener un manejo
                eficiente y permitir el acesso a varias personas conforme a peticiones.
                También esperaría sean muchos los datos que se planea almacenar, que 
                estos tengan una estructura bien definida y se relacionen entre sí.\\
                Cuando tengo una situación como la anterior no me conviene usar el sistema
                de almacenamiento del SO porque puede ser problemático organizar, editar
                y buscar datos, no tendría las ventajas del SMBD y proporcionar accesso a
                través de consultas a datos específicos también requeriría mucho trabajo.\\
                Cuando dejan de cumplirse parte de las condiciones planteadas al inicio, pueden
                existir alternativas viables que me permiten hacer lo que quiero sin tener que 
                pasar por todo el proceso de crear un SBD; más simples y versatiles.
                Por ejemplo, si sólo quiero hacer unas tablas para mi tarea, general algunas
                gráficas a partir de ellas y realizar unos calculos (promedios, sumas de
                secciones, etc.) Excel o alguna herramienta similar sería una buena opción.
            }
        \item {
            ¿Qué \textbf{ventajas}  y \textbf{desventajas} encuentras al 
            trabajar con una \textbf{base de datos}?
        }
            {
                Las ventajas de trabajar con una base de datos estan dadas por las garantías
                y características que le da el diseño y SMBD; integridad de datos, operacones 
                básicas, chequeo de redundancia, un sistema robusto capaz de crecer...\\
                En cuanto a desventajas, estas dependen de los límites en el diseño, por 
                ejemplo,  quiero guaradar un historial junto con los datos actuales (Almacenes) 
                o mi modelo no me permite representar lo que necesito) y el costo en tiempo y
                recursos de realizar una implementacion buena de mi BD. Sin contar la dificultad.
            }
        \item {
            Explica las diferencias entre los esquemas \textbf{externos}, 
            \textbf{conceptual} y \textbf{físico}. ¿Cómo se relacionan estos 
            conceptos con la \textbf{independencias física} y \textbf{lógica}? 
        }
            {
                El esquema externo depende unicamente del planteamiento de mi problema;
                representa los procesos de los usuarios que me interesa permitir.
                Una vez que se define se pasa al esquema conceptual, en el cual se busca 
                determinar como va a funcionar la base de datos en abstracto.
                El esquema físico finalmente se refiere a los detalles de la implementación; 
                metadatos, almacenamiento, etc.
                La independencia física marca el punto en el que se puede separar el nivel 
                físico de los otros aspectos de la base de datos (Nivel conceptual y externo).
                Asimismo, la independencia lógica separa el Nivel externo del resto de la
                infraestructura de la BD, ya que el uso que se le da a la base no debería
                relacionarse con la infraestructura de la BD; el usuario no  necesita saber.
            }
        \item {
            ¿Qué es el \textbf{diccionario de datos} y por qué es importante?
        }
            {
                Es la parte de la base de datos encargada de preservar el modo en el que se
                accede/describe a los datos. Desde el cómo se ven almacenados en la comp.
                hasta que nombre tiene la columna a la que pertenece este dato. Es decir,
                contiene a los metadatos que describen en general de la BD.
            }
        \item {
            Indica las principales caracterísitcas de alguno de los siguientes 
            modelos de bases de datos: \textbf{jerárquico, de red, orientado 
            a objetos}.
        }
        \item {
            Elabora una \textbf{línea de tiempo} en dónde indiques los principales
            hitos en el desarrollo de las bases de datos.
            }
        \item {
            Indica las responsabilidades que tiene un \textbf{Sistema Manejador
            de Bases de Datos} y para cada responsabilidad, explica los problemas
            que surgirían si dicha responsabilidad no se cumpliera.
        }
        \item {
            Supón que una pequeña compañía desea almacenar su información en una
            base de datos. Desea comprar la que tenga la menor cantidad de 
            caracterísitcas posibles, se desea ejecutar la aplicación en una 
            sólo computadora personal y no se planea compartir la información con
            nadie. Para cada una de las siguientes características, explica por
            qué se debería o no incluir en la base de datos que se desea comprar
            (suponiendo que se pieden comprar por separado): \textbf{seguridad, 
            control de concurrencia, recuperación en caso de fallos, lengua de 
            consulta, mecanismo de visitas, manejo de transacciones}.
        }
    \end{enumerate}

    \section{Invetigación}
    \begin{enumerate}[label=\alph*.]
        \item {
            ¿Qué es la \textbf{Calidad de Datos} y cómo se relaciona con las 
            bases de datos?
        }
        \item {
            Especifica las caracterísitcas más importantes de las bases de datos
            \textbf{NoSQL}, indica el modelo de datos que utilizan y sus 
            principales proveedores.
        }
        \item {
            ¿Qué es un \textbf{ORM}?
        }
    \end{enumerate}

    \section{Referencias}
\end{document}
