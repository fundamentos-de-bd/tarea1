\documentclass{article}

\usepackage[spanish]{babel}

\usepackage[margin = 1.5cm]{geometry}

\usepackage{enumitem}

\usepackage{hyperref}
\hypersetup{
    colorlinks=true,
    linkcolor=blue,
    filecolor=magenta,
    urlcolor=cyan,
}


\begin{document}
    \title{Fundamentos de Bases de Datos \\
        Tarea 1 \\
        Conceptos básicos} 
    \author{Díaz Gómez Silvia \\
    Eugenio Aceves Narciso Isaac \\
    Quiroz Castañeda Edgar}
    \date{18 de febrero del 2019}
    \maketitle

    \section{Conceptos generales}

    \begin{enumerate}[label=\alph*.]
        \item {
            ¿Porqué elegirías almacenar datos en un \textbf{sistema de bases 
            datos} en lugar de simplemente almacenarlos utilzizando el 
            \textbf{sistema de archivos} de un sistema operativo?¿En qué caso no
            tendría sentido utilizar un sistema de bases de datos?
        }
        \item {
            ¿Qué \textbf{ventajas}  y \textbf{desventajas} encuentras al 
            trabajar con una \textbf{base de datos}?
        }
        \item {
            Explica las diferencias entre los esquemas \textbf{externos}, 
            \textbf{conceptual} y \textbf{físico}. ¿Cómo se relacionan estos 
            conceptos con la \textbf{independencias física} y \textbf{lógica}? 
        }
        \item {
            ¿Qué es el \textbf{diccionario de datos} y por qué es importante?
        }
        \item {
            Indica las principales caracterísitcas de alguno de los siguientes 
            modelos de bases de datos: \textbf{jerárquico, de red, orientado 
            a objetos}.
            \begin{itemize}
                \item {Jerárquico\\
                Los datos se almacenan en una estructura de árbol, donde cada 
                dato tiene un padre, excepto un dato raíz, y posiblemente varios
                o ningún hijo. \\
                Funciona de forma similar al sistema de archivos de un sistema 
                operativo.
                \begin{itemize}
                    \item {Estructura \\
                    En general, es un árbol. La estructura de los subárbles no 
                    está limitada.
                    }
                    \item {Restricciones de integridad \\
                    Cada nodo tiene un sol dato. Se pueden definir los tipos de 
                    nodos posibles, cada nodo con atributos que a su vez son nodos.\\
                    Puede o no haber restricciones de unicidad, pero como para 
                    acceder a un dato hay que recorrer el árbol, esto puede ser 
                    difícil de mantener. \\
                    Tampoco hay restricciones de integridad, pues no es posible 
                    definir llaves foráneas al sólo poder tener un sólo padre. 
                    }
                    \item {
                    Operaciones \\
                    Las mismas operaciones que en un árbol. Notemos que como hay
                    padre únicos, no se pueden modelar de forma sencilla relaciones
                    de muchos a muchos. Además, para mantener el árbol, insertar 
                    un dato puede cambiar radicalmente toda la estructura, y para
                    realizar consultas hay que realizar recorridos por el árbol.
                    }
                \end{itemize}
                }
                \item {
                De red \\
                Dados los problemas del modelo jerárquico, se creó el modelo de 
                redes. En lugar de tener una estructura de árbol, se tiene una 
                una estructura más general de gráfica. Esto da más flexibilidad 
                en cuanto a las relaciones posibles.
                \begin{itemize}
                    \item {Estructura \\
                    Es una estructura de gráfica. Cada nodo puede tener varios 
                    padres e hijos}
                    \item {Restricciones de integridad \\
                    Igualmente que en el modelo jerárquico, se pueden definir los 
                    tipos de nodo. 
                    }
                    \item{Operaciones \\
                    Las mismas operaciones que en una gráfica. Aunque se puede
                    definir relaciones de muchos a muchos, y por lo tanto se 
                    pueden dar restricciones de integridad referencial, acceder a
                    un datos sigue requiriendo un recorrido de la gráfica, lo 
                    que dificulta mantener la integridad de unicidad.
                    }
                \end{itemize} 
                }
                \item {Orientado a objetos\\
                Los datos se modelan como objetos.
                \begin{itemize}
                    \item {Estructura\\
                    Cada dato es una instancia de un objeto, donde el estado de 
                    cada objeto (los atributos) son otros objetos.
                    \begin{itemize}
                        \item {Estructura\\
                        Son objetos}
                        \item {Restricciones de integridad \\
                        Las mismas que en la orientación a objetos. Hay 
                        restricciones de tipo, y si se manejan los atrributos 
                        como apuntadores, se tiene integridad referencial.
                        También se tiene el encapulamiento de datos}
                        \item{Operaciones \\
                        Son los métodos de los objetos.
                        }
                    \end{itemize}}
                \end{itemize}}
            \end{itemize}
        }
        \item {
            Elabora una \textbf{línea de tiempo} en dónde indiques los principales
            hitos en el desarrollo de las bases de datos.
            }
        \item {
            Indica las responsabilidades que tiene un \textbf{Sistema Manejador
            de Bases de Datos} y para cada responsabilidad, explica los problemas
            que surgirían si dicha responsabilidad no se cumpliera. \\
            Un SMBD tiene como responsabilidades
            \begin{itemize}
                \item {Definir tipos, estructuras y restricciones. De no 
                funcionar bien, ni siquiera se podría hablar de integridad, pues
                no se podría definir ningún tipo de regla u estructura correcta.}
                \item {Construir la parte física de la base de datos. De no 
                realizarse correctamente, se podrían perder o corromper los datos}
                \item {Manipular los datos, como en funciones u operaciones. De
                no fucnionar adecuadamente, podría pasar que datos perfectamente
                válidos pierdan su integridad(sean alterados) después de ser 
                manipulados}
                \item {Compartir la base de datos en diferentes vistas. De no 
                funcionar bien, los datos a los que se acceden pueden ser erróneos,
                a pesar de que internamente la base de datos funcione 
                perfectamente}
            \end{itemize}
        }
        \item {
            Supón que una pequeña compañía desea almacenar su información en una
            base de datos. Desea comprar la que tenga la menor cantidad de 
            caracterísitcas posibles, se desea ejecutar la aplicación en una 
            sólo computadora personal y no se planea compartir la información con
            nadie. Para cada una de las siguientes características, explica por
            qué se debería o no incluir en la base de datos que se desea comprar
            (suponiendo que se pieden comprar por separado): \textbf{seguridad, 
            control de concurrencia, recuperación en caso de fallos, lengua de 
            consulta, mecanismo de visitas, manejo de transacciones}.
            \begin{itemize}
                \item {Seguridad \\
                Esto es inclusive independiente del tipo de sistema que se 
                tenga. La seguridad es indispensable, y más aún al ser un 
                negocio, pues manejan información delicada acerca de clientes,
                proveedores, empleados, entre otros.}
                \item {Control de concurrencia \\
                Cómo no se planea tener múltiples usuarios, entonces probablemente
                no hays muchos problemas de concurrencia, por lo que no es 
                necesario tener un control de ella}
                \item {Recuperación en caso de fallos \\
                Como toda la información estará en una sola computadora, 
                entonces es de suma importancia pretefer la información es esa 
                punica computadora. Por lo que la recuperación de fallos es 
                indispensable}
                \item {Lengua de consulta \\
                Esto es independiente de la cantidad de usuarios o de 
                computadoras, además de indispensable, pues no se podría 
                manipular la onfromación sin ella}
                \item {Mecanismo de visitas \\
                Cómo no se planea tener visitas (pues sólo habrá un usuario), 
                entonces no es necesario contar con un mecanismo de visitas.
                }
                \item {Manejo de transacciones \\
                Esto también es independiente de la cantidad de usuarios o de la
                cantidad de computadoras. Sin esto, no se podría garantizar la 
                intergridad de los datos, por lo que es indispensable}
            \end{itemize}
        }
    \end{enumerate}

    \section{Invetigación}
    \begin{enumerate}[label=\alph*.]
        \item {
            ¿Qué es la \textbf{Calidad de Datos} y cómo se relaciona con las 
            bases de datos?
        }
        \item {
            Especifica las caracterísitcas más importantes de las bases de datos
            \textbf{NoSQL}, indica el modelo de datos que utilizan y sus 
            principales proveedores.
        }
        \item {
            ¿Qué es un \textbf{ORM}?
        }
    \end{enumerate}

    \section{Referencias}
    \begin{thebibliography}{}
        \bibitem[Gil15]{Gil15}
            Gilfillan, I. (2015, June 8). Exploring Early Database Models. 
            Retrieved February 17, 2019, from 
            \href{https://mariadb.com/kb/en/library/exploring-early-database-models/}
            {https://mariadb.com/kb/en/library/exploring-early-database-models/}
    \end{thebibliography}
\end{document}