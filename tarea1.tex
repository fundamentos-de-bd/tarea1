\documentclass{article}

\usepackage[spanish]{babel}

\usepackage[margin = 1.5cm]{geometry}

\usepackage{enumitem}


\begin{document}
    \title{Fundamentos de Bases de Datos \\
        Tarea 1 \\
        Conceptos básicos} 
    \author{Díaz Gómez Silvia \\
    Eugenio Aceves Narciso Isaac \\
    Quiroz Castañeda Edgar}
    \date{18 de febrero del 2019}
    \maketitle

    \section{Conceptos generales}

    \begin{enumerate}[label=\alph*.]
        \item {
            ¿Porqué elegirías almacenar datos en un \textbf{sistema de bases 
            datos} en lugar de simplemente almacenarlos utilzizando el 
            \textbf{sistema de archivos} de un sistema operativo?¿En qué caso no
            tendría sentido utilizar un sistema de bases de datos?
        }
        \item {
            ¿Qué \textbf{ventajas}  y \textbf{desventajas} encuentras al 
            trabajar con una \textbf{base de datos}?
        }
        \item {
            Explica las diferencias entre los esquemas \textbf{externos}, 
            \textbf{conceptual} y \textbf{físico}. ¿Cómo se relacionan estos 
            conceptos con la \textbf{independencias física} y \textbf{lógica}? 
        }
        \item {
            ¿Qué es el \textbf{diccionario de datos} y por qué es importante?
        }
        \item {
            Indica las principales caracterísitcas de alguno de los siguientes 
            modelos de bases de datos: \textbf{jerárquico, de red, orientado 
            a objetos}.
        }
        \item {
            Elabora una \textbf{línea de tiempo} en dónde indiques los principales
            hitos en el desarrollo de las bases de datos.
            }
        \item {
            Indica las responsabilidades que tiene un \textbf{Sistema Manejador
            de Bases de Datos} y para cada responsabilidad, explica los problemas
            que surgirían si dicha responsabilidad no se cumpliera.
        }
        \item {
            Supón que una pequeña compañía desea almacenar su información en una
            base de datos. Desea comprar la que tenga la menor cantidad de 
            caracterísitcas posibles, se desea ejecutar la aplicación en una 
            sólo computadora personal y no se planea compartir la información con
            nadie. Para cada una de las siguientes características, explica por
            qué se debería o no incluir en la base de datos que se desea comprar
            (suponiendo que se pieden comprar por separado): \textbf{seguridad, 
            control de concurrencia, recuperación en caso de fallos, lengua de 
            consulta, mecanismo de visitas, manejo de transacciones}.
        }
    \end{enumerate}

    \section{Invetigación}
    \begin{enumerate}[label=\alph*.]
        \item {
            ¿Qué es la \textbf{Calidad de Datos} y cómo se relaciona con las 
            bases de datos?
        }
        \item {
            Especifica las caracterísitcas más importantes de las bases de datos
            \textbf{NoSQL}, indica el modelo de datos que utilizan y sus 
            principales proveedores.
        }
        \item {
            ¿Qué es un \textbf{ORM}?
        }
    \end{enumerate}

    \section{Referencias}
\end{document}