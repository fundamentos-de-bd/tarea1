\documentclass[10pt]{article}
\usepackage[utf8]{inputenc}
\usepackage[spanish]{babel}
\usepackage[usenames,dvipsnames,svgnames,table]{xcolor}
\usepackage{multirow}
\usepackage{diagbox}
\usepackage{booktabs}
\usepackage{anysize} 
\usepackage{hyperref}
\usepackage{helvet}
\renewcommand\refname{Referencias}
\marginsize{2cm}{2cm}{2.0cm}{2cm}
\usepackage{enumitem}
\usepackage{setspace}


\hypersetup{
    colorlinks=true,
    linkcolor=blue,
    filecolor=magenta,
    urlcolor=cyan,
    citecolor=blue
}


\usepackage{chronosys}



\begin{document}
    \title{Fundamentos de Bases de Datos \\
        Tarea 1 \\
        Conceptos básicos} 
    \author{}
    \date{18 de febrero del 2019}
    \maketitle
    
    \section{Conceptos generales}\vspace{0.5cm}

        \begin{enumerate}[label=\alph*.]
    	\item  {¿Porqué elegirías almacenar datos en un \textbf{sistema de bases 
    			datos} en lugar de simplemente almacenarlos utilzizando el 
    		\textbf{sistema de archivos} de un sistema operativo?¿En qué caso no
    		tendría sentido utilizar un sistema de bases de datos?
    	}\\
    	{
    	   Elegiría el SBD si vale la pena la inversión en tiempo y     recursos en todo el diseño e infraestructura de la relacionados. Es decir, si mis 
    		necesidades involucran preservar la integridad de los datos, tener un manejo
    		eficiente y permitir el acesso a varias personas conforme a peticiones.
    		También esperaría sean muchos los datos que se planea almacenar, que 
    		estos tengan una estructura bien definida y se relacionen entre sí.\\
    		Cuando tengo una situación como la anterior no me conviene usar el sistema
    		de almacenamiento del SO porque puede ser problemático organizar, editar
    		y buscar datos, no tendría las ventajas del SMBD y proporcionar accesso a
    		través de consultas a datos específicos también requeriría mucho trabajo.\\
    		Cuando dejan de cumplirse parte de las condiciones planteadas al inicio, pueden
    		existir alternativas viables que me permiten hacer lo que quiero sin tener que 
    		pasar por todo el proceso de crear un SBD; más simples y versatiles.
    		Por ejemplo, si sólo quiero hacer unas tablas para mi tarea, general algunas
    		gráficas a partir de ellas y realizar unos calculos (promedios, sumas de
    		secciones, etc.) Excel o alguna herramienta similar sería una buena opción.
    	}\\
    
    	\item {
    		¿Qué \textbf{ventajas}  y \textbf{desventajas} encuentras al 
    		trabajar con una \textbf{base de datos}?\\
    	}
    	{
    		 Las ventajas de trabajar con una base de datos estan dadas por las garantías
    		y características que le da el diseño y SMBD; integridad de datos, operacones 
    		básicas, chequeo de redundancia, un sistema robusto capaz de crecer...\\
    		En cuanto a desventajas, estas dependen de los límites en el diseño, por 
    		ejemplo,  quiero guaradar un historial junto con los datos actuales (Almacenes) 
    		o mi modelo no me permite representar lo que necesito) y el costo en tiempo y
    		recursos de realizar una implementacion buena de mi BD. Sin contar la dificultad.
    	}\\
    
    	\item {
    		Explica las diferencias entre los esquemas \textbf{externos}, 
    		\textbf{conceptual} y \textbf{físico}. ¿Cómo se relacionan estos 
    		conceptos con la \textbf{independencias física} y \textbf{lógica}? \\
    	}
    	{
    		 El esquema externo depende unicamente del planteamiento de mi problema;
    		representa los procesos de los usuarios que me interesa permitir.
    		Una vez que se define se pasa al esquema conceptual, en el cual se busca 
    		determinar como va a funcionar la base de datos en abstracto.
    		El esquema físico finalmente se refiere a los detalles de la implementación; 
    		metadatos, almacenamiento, etc.
    		La independencia física marca el punto en el que se puede separar el nivel 
    		físico de los otros aspectos de la base de datos (Nivel conceptual y externo).
    		Asimismo, la independencia lógica separa el Nivel externo del resto de la
    		infraestructura de la BD, ya que el uso que se le da a la base no debería
    		relacionarse con la infraestructura de la BD; el usuario no  necesita saber.
    	}\\
    
    	\item {
    		¿Qué es el \textbf{diccionario de datos} y por qué es importante?\\
    	}
    	{
    	    Es la parte de la base de datos encargada de preservar el modo en el que se
    		accede/describe a los datos. Desde el cómo se ven almacenados en la comp.
    		hasta que nombre tiene la columna a la que pertenece este dato. Es decir,
    		contiene a los metadatos que describen en general de la BD.
    	}\\
    
        \item {
            Indica las principales caracterísitcas de alguno de los siguientes 
            modelos de bases de datos: \textbf{jerárquico, de red, orientado 
            a objetos}.
       
            \begin{itemize}
                \item {Jerárquico\\
                Los datos se almacenan en una estructura de árbol, donde cada 
                dato tiene un padre, excepto un dato raíz, y posiblemente varios
                o ningún hijo. \\
                Funciona de forma similar al sistema de archivos de un sistema 
                operativo.
                \begin{itemize}
                    \item {Estructura \\
                    En general, es un árbol. La estructura de los subárbles no 
                    está limitada.
                    }
                    \item {Restricciones de integridad \\
                    Cada nodo tiene un sol dato. Se pueden definir los tipos de 
                    nodos posibles, cada nodo con atributos que a su vez son nodos.\\
                    Puede o no haber restricciones de unicidad, pero como para 
                    acceder a un dato hay que recorrer el árbol, esto puede ser 
                    difícil de mantener. \\
                    Tampoco hay restricciones de integridad, pues no es posible 
                    definir llaves foráneas al sólo poder tener un sólo padre. 
                    }
                    \item {
                    Operaciones \\
                    Las mismas operaciones que en un árbol. Notemos que como hay
                    padre únicos, no se pueden modelar de forma sencilla relaciones
                    de muchos a muchos. Además, para mantener el árbol, insertar 
                    un dato puede cambiar radicalmente toda la estructura, y para
                    realizar consultas hay que realizar recorridos por el árbol.
                    }
                \end{itemize}
                }
                \item {
                De red \\
                Dados los problemas del modelo jerárquico, se creó el modelo de 
                redes. En lugar de tener una estructura de árbol, se tiene una 
                una estructura más general de gráfica. Esto da más flexibilidad 
                en cuanto a las relaciones posibles.
                \begin{itemize}
                    \item {Estructura \\
                    Es una estructura de gráfica. Cada nodo puede tener varios 
                    padres e hijos}
                    \item {Restricciones de integridad \\
                    Igualmente que en el modelo jerárquico, se pueden definir los 
                    tipos de nodo. 
                    }
                    \item{Operaciones \\
                    Las mismas operaciones que en una gráfica. Aunque se puede
                    definir relaciones de muchos a muchos, y por lo tanto se 
                    pueden dar restricciones de integridad referencial, acceder a
                    un datos sigue requiriendo un recorrido de la gráfica, lo 
                    que dificulta mantener la integridad de unicidad.
                    }
                \end{itemize} 
                }
                \item {Orientado a objetos\\
                Los datos se modelan como objetos. Cada dato es una instancia 
                de un objeto, donde el estado de cada objeto (los atributos) 
                son otros objetos.
                    
                \begin{itemize}
                    \item {Estructura\\
                    Son objetos}
                    \item {Restricciones de integridad \\
                    Las mismas que en la orientación a objetos. Hay 
                    restricciones de tipo, y si se manejan los atrributos 
                    como apuntadores, se tiene integridad referencial.
                    También se tiene el encapulamiento de datos}
                    \item{Operaciones \\
                    Son los métodos de los objetos.\\
                    }
                \end{itemize}
                }
            \end{itemize}
        }
    
        \item {
            Elabora una \textbf{línea de tiempo} en dónde indiques los principales
            hitos en el desarrollo de las bases de datos.
            \startchronology[startyear=1959, stopyear=1980]
            \chronoevent{1966}{IBM IMS}
            \chronoevent{1969}{MR}
            \chronoevent{1971}{CODASYL}
            \chronoevent{1974}{Ingres}
            \chronoevent{1976}{E/R}
            \chronoevent{1977}{SysR}
            \stopchronology
            }
            \startchronology[startyear=1981, stopyear=2000]
            \chronoevent{1985}{OO}
            \chronoevent{1986}{SQL}
            \chronoevent{1995}{Internet}
            \chronoevent{1997}{XML}
            \stopchronology
            Donde 
            \begin{itemize}
                \item {IBM IMS: IBM Information Managment System}
                \item {MR: Modelo relacional, propuesto por E. Codd}
                \item {CODASYL: Estandar del modelo de redes, definido en la 
                conferencia CODASYL de 1971}
                \item {Ingres: origen de Postgres}
                \item {Modelo Entidad/Relación}
                \item {System R: origen de IBM DB2}
                \item {Modelo Orientado a Objetos}
                \item {SQL: Standard Query Language se definice com estándar en 
                1971 por la INS y la ANSI}
                \item {Primeras aplicaciones de internet que acceen a bases de
                datos}
                \item {XML: se utiliza para procesar las bases de datos para 
                resolver problemas varios}\\
            \end{itemize}
        
        \item {
            Indica las responsabilidades que tiene un \textbf{Sistema Manejador
            de Bases de Datos} y para cada responsabilidad, explica los problemas
            que surgirían si dicha responsabilidad no se cumpliera. \\
            Un SMBD tiene como responsabilidades
            \begin{itemize}
                \item {Definir tipos, estructuras y restricciones. De no 
                funcionar bien, ni siquiera se podría hablar de integridad, pues
                no se podría definir ningún tipo de regla u estructura correcta.}
                \item {Construir la parte física de la base de datos. De no 
                realizarse correctamente, se podrían perder o corromper los datos}
                \item {Manipular los datos, como en funciones u operaciones. De
                no fucnionar adecuadamente, podría pasar que datos perfectamente
                válidos pierdan su integridad(sean alterados) después de ser 
                manipulados}
                \item {Compartir la base de datos en diferentes vistas. De no 
                funcionar bien, los datos a los que se acceden pueden ser erróneos,
                a pesar de que internamente la base de datos funcione 
                perfectamente}\\
            \end{itemize}
        }
    
        \item {
            Supón que una pequeña compañía desea almacenar su información en una
            base de datos. Desea comprar la que tenga la menor cantidad de 
            caracterísitcas posibles, se desea ejecutar la aplicación en una 
            sólo computadora personal y no se planea compartir la información con
            nadie. Para cada una de las siguientes características, explica por
            qué se debería o no incluir en la base de datos que se desea comprar
            (suponiendo que se pieden comprar por separado): \textbf{seguridad, 
            control de concurrencia, recuperación en caso de fallos, lengua de 
            consulta, mecanismo de visitas, manejo de transacciones}.
            \begin{itemize}
                \item {Seguridad \\
                Esto es inclusive independiente del tipo de sistema que se 
                tenga. La seguridad es indispensable, y más aún al ser un 
                negocio, pues manejan información delicada acerca de clientes,
                proveedores, empleados, entre otros.}
                \item {Control de concurrencia \\
                Cómo no se planea tener múltiples usuarios, entonces probablemente
                no hays muchos problemas de concurrencia, por lo que no es 
                necesario tener un control de ella}
                \item {Recuperación en caso de fallos \\
                Como toda la información estará en una sola computadora, 
                entonces es de suma importancia pretefer la información es esa 
                punica computadora. Por lo que la recuperación de fallos es 
                indispensable}
                \item {Lengua de consulta \\
                Esto es independiente de la cantidad de usuarios o de 
                computadoras, además de indispensable, pues no se podría 
                manipular la onfromación sin ella}
                \item {Mecanismo de visitas \\
                Cómo no se planea tener visitas (pues sólo habrá un usuario), 
                entonces no es necesario contar con un mecanismo de visitas.
                }
                \item {Manejo de transacciones \\
                Esto también es independiente de la cantidad de usuarios o de la
                cantidad de computadoras. Sin esto, no se podría garantizar la 
                intergridad de los datos, por lo que es indispensable}\\
            \end{itemize}
        }
    \end{enumerate}

    \section{Invetigación} \vspace{0.5cm}
    \begin{enumerate}[label=\alph*.]
        \item {
            ¿Qué es la \textbf{Calidad de Datos} y cómo se relaciona con las 
            bases de datos?}\\
        La calidad de datos se refiere a los
            procesos y técnicas enfocadas a mejorar la eficacia de los datos existentes en nuestras bases de
            datos. En este sentido, para que un proceso de calidad de datos sea realmente eficaz, éste debería
            ser repetible y fácil de entender, de manera que permitiera generar un proceso que se vuelva un
            ciclo de mejora y que cada vez que fuera ejecutado generara datos con mayor calidad, permitiendo
            desarrollar reportes para dar seguimiento a los progresos y proporcionar la mejora continua de la
            calidad de los datos. Por lo tanto, La calidad de datos esta relacionada con las bases de datos justo porque se enfoca en mejorar la eficacia de los datos existentes en una base de datos, es decir, que los datos se mantengan en el mejor estado posible, que no haya duplicidades, que sean correctos, que esten actualizados de forma continua, etc. \cite[pag. 6]{DQ} \\
            
            \item Especifica las características más importantes de las bases de datos \textbf{NoSQL}, indica el modelo
            de datos que utilizan y principales proveedores.\\
            Es un modelo no relacional, sus principales características son:
            \begin{itemize}
            	\item Proporcionan un rendimiento superior al que ofrecen los sistemas RDBS convencionales.
            	\item Pueden soportar lenguajes de consulta de tipo SQL.
            	\item Se ejecutan en máquinas con pocos recursos. 
            	\item Pueden manejar gran cantidad de datos.
            	\item Ofrecen operaciones bastante simples sobre estructuras de datos flexibles. De hecho, por lo general, aprovechan esta simplicidad para proporcionar una alta escalabilidad y un rendimiento masivo.
            	\item Manejan elementos individuales, identificados por claves únicas.
            	\item Se clasifican de acuerdo a su forma de almacenar los datos: almacenes de registros extensibles, almacenes de documentos y almacenes de valor clave.
            \end{itemize}
            
            Principales proveedores: Hypertable, Cassandra, MongoDB, DynamoDB, HBase y Redis.\\
            
            \item ¿Qué es un \textbf{ORM}?\\
            El mapeo objeto-relacional que se conoce comunmente como ORM por sus siglas en ingles Object Relational Mapping, es una tecnica de programación para convertir datos entre el sistema de tipos  utilizando un lenguaje de programación orientado a objetos y la utilizacion de una base de datos relacional. Es una tecnica que se utiliza para poder ligar las bases de datos  y los conceptos de orientación a objetos creando "base de datos virtuales".   Esta
            transformación permite el uso de las bondades de la programación orientada a objetos como son la herencia y polimorfismo. \cite{orm}\\
        \end{enumerate}

  
    \begin{thebibliography}{}
        \bibitem{Gil15}
            Gilfillan, I. (2015, June 8). Exploring Early Database Models. 
            Retrieved February 17, 2019, from 
            \href{https://mariadb.com/kb/en/library/exploring-early-database-models/}
            {https://mariadb.com/kb/en/library/exploring-early-database-models/}
            \bibitem{DQ} PowerData.Calidad de Datos. URL: \url{https://landings.powerdata.es/hubfs/POWERDATA_TOFU_-_calidad_de_datos.pdf}
            \bibitem{nosql} Paolo Atzenin, Uniform access to NoSQL systems. Information Systems,43, July 2014, Pages 117-133.
            \bibitem{nosql2}acens, the cloud services company. Bases de datos NoSQL. Qué son y tipos que nos podemos encontrar. URL: \url{https://www.acens.com/wp-content/images/2014/02/bbdd-nosql-wp-acens.pdf}
            \bibitem{orm} \url{https://en.wikipedia.org/wiki/Object-relational_mapping}
    \end{thebibliography}
\end{document}